%\title{Creating an index with the makeidx package}
% Example from http://www.dickimaw-books.com/latex/thesis/html/makeidx.html

\documentclass[12pt,oneside]{scrbook} 

\usepackage{makeidx} 
\usepackage{minted}
\usepackage[french]{babel}
\usepackage{graphicx}
\selectlanguage{french}
\usepackage[T1]{fontenc}
\usepackage[utf8]{inputenc}
\usepackage{xcolor}
\definecolor{bg}{rgb}{0.95,0.95,0.95}
\definecolor{bg_js}{rgb}{0.95,0.55,0.55}
\newminted{typescript}{javascript,bgcolor=bg}

\makeindex 

\title{Electron} 
\author{Me}
\date{\today}

\begin{document} 
\maketitle 
\tableofcontents


%\includegraphics{util_inherits.png}

%%%%%%%%%%%%%%NPM/Node
\chapter{NPM/Node}
\section{Commandes}
\begin{itemize}
\item Initialisation d'un projet\\
La commande \textit{init} permet de créer un projet .json vide
\item Installation d'un module\\
Dans un projet existant, on peut rajouter un module et en même temps la dépendance:
\begin{minted}[bgcolor=bg]{shell}
npm install "module" -D
\end{minted}
\item Ajouter automatiquement ce module en dépendance de développement
\begin{minted}[bgcolor=bg]{shell}
npm install "module" -D
\end{minted}
\item Ajouter automatiquement ce module en dépendance de livraison
\begin{minted}[bgcolor=bg]{shell}
npm install "module" --save
\end{minted}
\item Lancer un script\\
Dans la liste des scripts du fichier .json, il peut être possible d'en lancer un:
\begin{minted}[bgcolor=bg]{shell}
npm run es6
\end{minted}
Dans cet exemple:
\begin{minted}[bgcolor=bg]{shell}
"scripts": {
    "es6": "./node_modules/.bin/http-server -c-1 ."
  }
\end{minted}
\item Débogage dans Chrome\\
On peut lancer un debogage d'un fichier JS depuis Node. Celui-ci lance un browker permettant par la suite de deboguer depuis les devtools de Chrome gràce au lien renvoyé par cette commande:
\begin{minted}[bgcolor=bg]{shell}
node --debug-brk=35220 --inspect samples\learning-types.js
\end{minted}
\end{itemize}


\backmatter 

\printindex 

\end{document} 